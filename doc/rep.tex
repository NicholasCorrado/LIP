        % % Title and author(s)
%%%%%%%%%%%%%%%%%%%%%%%%%%%%%%%%%%%%%%%%%%%%%%%%%%%%%%%
\title{Accelerating Joins with Filters}
\author{Nicholas Corrado \and Xiating Ouyang}
\date{}
%%%%%%%%%%%%%%%%%%%%%%%%%%%%%%%%%%%%%%%%%%%%%%%%%%%%%%%
\documentclass[10pt]{article}
%%%%%%%%%%%%%%%%%%%%%%%%%%%%%%%%%%%%%%%%%%%%%%%%%%%%%%%
% %
% % The next command allows your in import encapsulated
% % postscript files, .epsf or .eps files, which
% % contain vector graphic image data.
% %
%%%%%%%%%%%%%%%%%%%%%%%%%%%%%%%%%%%%%%%%%%%%%%%%%%%%%%%
\usepackage{graphicx}
\usepackage{charter,eulervm}
\usepackage{simpleConference}
%\renewcommand{\baselinestretch}{1.5}
\setcounter{secnumdepth}{3} % default value for 'report' class is "2"
\usepackage{amsthm,amsmath,amssymb,upgreek,marvosym,mathtools}
\usepackage{array}
\usepackage{makeidx}  % allows for indexgeneration
\usepackage{paralist}
\usepackage{subfig}
\usepackage{tabularx}
\usepackage{tabu}
\usepackage{comment}
\usepackage[nottoc]{tocbibind}
\usepackage[usenames,dvipsnames]{color}
\usepackage[pdftex,breaklinks,colorlinks,citecolor={blue}, linkcolor={blue},urlcolor=Maroon]{hyperref}
\usepackage{tkz-graph}
 \linespread{1.25}
\usetikzlibrary{automata, positioning,arrows,shapes,decorations.pathmorphing}

 \tikzset{
->, % makes the edges directed
>=stealth, % makes the arrow heads bold
node distance=3cm, % specifies the minimum distance between two nodes. Change if necessary.
every state/.style={thick, fill=gray!10}, % sets the properties for each ’state’ node
initial text=$ $, % sets the text that appears on the start arrow
}

\newtheorem{theorem}{Theorem}[section]
\newtheorem{lemma}{Lemma}[section]
\newtheorem{reduction}{Reduction}[section]
\newtheorem{proposition}{Proposition}[section]
\newtheorem{scolium}{Scolium}[section]   %% And a not so common one.
\newtheorem{definition}{Definition}[section]
\newtheorem{conjecture}{Conjecture}[section]
\newtheorem{corollary}{Corollary}[section]
%\newenvironment{proof}{{\sc Proof:}}{~\hfill QED}
\newenvironment{AMS}{}{}
\newenvironment{keywords}{}{}
\DeclarePairedDelimiter{\norm}{\lVert}{\rVert}
\newcommand{\todo}{(TO BE CONTINUED...)}
\graphicspath{ {./pics/} }

\newcommand{\paris}[1]{{\color{blue} Paris: [{#1}]}}

\newcommand{\xiating}[1]{{\color{blue} Xiating: [{#1}]}}


\newcommand{\trans}[1]{
	#1^\mathsf{T}
}

\newcommand{\db}{$\mathbf{db}$}
\newcommand{\sjfq}{\texttt{sjfCQA}}
\newcommand{\bcq}{\texttt{bcq}}
\newcommand{\problem}[1]{\textsc{certainty}($#1$)}
\newcommand{\FO}{$\mathbf{FO}$}
\newcommand{\PTIME}{$\mathbf{P}$}
\newcommand{\LSPACE}{$\mathbf{L}$}
\newcommand{\coNP}{$\mathbf{coNP}$}
\newcommand{\und}[1]{\underline{#1}}
\newcommand{\NL}{$\mathbf{NL}$}
\newcommand{\JOIN}{\bowtie}

\begin{document}
\newpage
\maketitle


\abstract{In query optimization on star schemas, lookahead information passing (LIP) is a strategy exploiting the efficiency of probing succinct filters to eliminate practically all facts that do not appear in the final join results before performing the actual join. Assuming data independency across all columns in the fact table, LIP achieves efficient and robust query optimization. We present LIP-$k$, a variant of LIP that only remembers the hit/miss statistics for the previous $k$ batches, achieving empirically efficient query execution on fact table with correlated and even adversarial data columns. We implemented LIP and LIP-$k$ on a skeleton database on top of Apache Arrow and analyze the performance of each variant of LIP using the notion of competitive ratio in online algorithms.}


\section{Introduction}

Performing join operations in database management systems using Star Schemas is a fundamental and prevalent task in the modern data industry. Continuous efforts have been spent on building a reliable query optimizer over the last few decades. However, the current optimizers may still produce disastrously inefficient query plans which involves processing unnecessarily gigantic intermediate tables \cite{leis2015good,rabl2013variations}. The \textit{lookahead Information Passing (LIP)} strategy aggressively uses Bloom Filters to filter the fact tables to effectively reduce the sizes of the intermediate tables, provably as efficient and robust as computing the join using the optimal query plan \cite{zhu2017looking}. The key idea behind LIP is to estimate the filter selectivity of each dimension table and adaptively reorder the sequence of applying the filters to the fact table. 


The filtering process can be modeled as the \textsc{LIP} problem in an online setting: Suppose we fix $n$ filters, and the tuples in the fact table arrives in an online fashion. Upon arrival of each tuple, one has to decide a sequence of filters to probe the tuple, with an objective of minimizing the number of probes needed to decide whether to accept the tuple and forward it to the hash join phase, or to eliminate it. A mechanism deciding the sequence of applying the filters is thus crucial to the success of LIP. If a tuple passes all filters, \textit{all} mechanisms have to probe the tuple to all filters to confirm its passage; and if a tuple is eliminated by the filters, the \textit{optimal} mechanism would apply any filter that rejects the tuple in the first place, using only one probe. Thus given any fact table $F$, the number of probes that an optimal mechanism requires to process all tuples in the fact table can be readily computed: 
\[
	\textsc{OPT}(F) = n|F_{\text{pass}}| + |F_{\text{reject}}|,
\]
where $|F_{\text{pass}}|$ and $|F_{\text{reject}}|$ are the number of tuples in $F$ that pass all filters and are rejected in $F$ respectively. For any mechanism $\mathcal{M}$, denoted by \textsc{ALG}$_{\mathcal{M}}(F)$ the number of probes that $\mathcal{M}$ performed to process all tuples in the fact table.  The performance of any mechanism $\mathcal{M}$ can thus be measured by 
\[
	\max_{F}\frac{\textsc{ALG}_{\mathcal{M}}(F)}{\textsc{OPT}(F)},
\]
called the \textit{competitive ratio} of $\mathcal{M}$. The competitive ratio is always at least 1 by definition, and in this problem the competitive ratio is at most $n$, the number of filters, since one mechanism can probe each tuple to at most $n$ filters.


This project aims at designing efficient LIP mechanisms and measure their performance in terms of their overall running time and competitive ratio. We will build a skeleton database system on top of Apache Arrow supporting LIP and hash-joins to conduct our experiments and test the performance of our variant LIP mechanisms against the hash-join and the orignal LIP. We also present a theoretical result showing that no deterministic mechanism can have a competitive ratio better than $n$, and discuss possible extensions of LIP to use randomness to design a better mechanism.


%Experimental results show that as LIP uses more filters, the increase in performance improvement diminishes, displaying a concave curve. One reason is that if the cache cannot hold all filters, then evicting the filters causes significant overhead in the performance. Moreover, when probing each fact tuple against all filters, the more selective filters are inserted to cache first. When the cache is full, the cache manager has to evict a more selective filter so that a less selective filter can be inserted to cache. Once LIP switches to probing the next fact tuple, the more selective filter is inserted into the cache again, while one could have skipped probing against certain inselective filters to reduce the replacement overhead.


%This project aims at investigating the effect of skipping certain LIP filter on improving the performance of LIP, and if possible, derive a theoretical guarantee on the performance of LIP against the optimal joining sequence. We will build a skeleton database system on top of Apache Arrow supporting LIP and hash-joins to conduct our experiments and test the performance of our revised LIP strategy against the hash-join.




\section{Lookahead Information Passing (LIP)}
In this section, we first present the LIP strategy in \cite{zhu2017looking}. We then discuss our variant LIP-$k$, designed to respond to local skewness more quickly than LIP. Finally, we discuss the competitive ratios of all deterministic mechanisms and provide proof on its lower bound. Possible extensions of LIP using randomness is also discussed.

\subsection{LIP}
The LIP strategy has three stages: (1) Build a hash table and a filter for each dimension table, (2) probe each fact tuple on the filters, producing a set of fact tuples with false positives, and (3) probe the hash table of each dimension table to eliminate the false positives. In what follows, we mainly discuss stage (2), since stages (1) and (3) are readily implemented by the filter constructors and database engine, respectively.

Let $F$ be the fact table and $D_i$ the dimension tables for $1 \leq i \leq n$. We denote the number of facts in $F$ and each $D_i$ as $|F|$ and $|D_i|$. Each LIP filter on $D_i$ is a Bloom filter with false positive rate $\varepsilon$. The true selectivity $\sigma_i$ of $D_i$ on fact table $F$ is given by $\sigma_i = |D_i \JOIN_{pk_i = fk_i} F| / |F|,$ where $pk_i$ is the primary key of $D_i$ and $fk_i$ is the foreign key of $D_i$ in $F$. The \texttt{LIP-join} algorithm, depicted in Figure \ref{fig:lip}, computes the indices of tuples in $F$ that pass all LIP filters. Note that because of the false positive rate $\varepsilon$ associated with each filter, the set of indices is a superset of the true set of indices of tuples appearing in the final join result.

The partition in \cite{zhu2017looking} satisfies that $|F_{t+1}| = 2|F_{t}|$ at line 5, and the algorithm approximates the true selectiveness $\sigma_i$ of each dimension $D_i$ using $pass[i]/count[i]$, the aggregated selectiveness since the beginning.

\begin{figure*}[h!]
	\centering
	\tikz\path (0,0) node[draw, text width=.8\textwidth, rectangle, inner xsep=20pt, inner ysep=10pt]{
		\begin{minipage}[t!]{\textwidth}
			{\sc Procedure}: \texttt{LIP-join}
			\\
			{\sc Input}: a fact table $F$ and a set of $n$ Bloom filters $f_i$ for each $D_i$ with $1 \leq i \leq n$
 			\\
			{\sc Output}: Indices of tuples in $F$ that pass the filtering
			\begin{tabbing}
				Aaa\=aaA\=Aaa\=Aaa\=Aaa\=AAAAAAAAAAAAAAAAAAAAAAAAA\=A \kill
				1.\> Initialize $I = \emptyset$
				\\
				2.\> {\bf foreach } filter $f$ {\bf do}
				\\
				3.\>\> $count[f] \leftarrow 0$
				\\
				4.\>\> $pass[f] \leftarrow 0$ 
				\\
				5.\> Partition $F = \bigcup_{1 \leq t \leq T}F_t$. 
				\\
				6.\> {\bf foreach } fact block $F_t$ {\bf do} 
				\\
				7.\>\> {\bf foreach } filter $f$ in order {\bf do}
				\\
				8.\>\>\> {\bf foreach} index $j \in F_t$ {\bf do}
				\\
				9.\>\>\>\> $count[f] \leftarrow count[f] + 1$
				\\
				10.\>\>\>\> {\bf if }$f$ contains $F_t[j]$ 
				\\
				11.\>\>\>\>\> $I \leftarrow I \cup \{j\}$ 
				\\
				12.\>\>\>\>\> $pass[f] \leftarrow pass[f] + 1$
				\\
				13.\>\> {\bf sort} filters $f$ in nondesending order of $pass[f]/count[f]$
				\\
				14.\> {\bf return } $I$
			\end{tabbing}  
		\end{minipage}
	};
	\caption{The LIP algorithm for computing the joins.}
	\label{fig:lip}
\end{figure*}


\subsection{LIP-$k$}

The LIP strategy in Figure \ref{fig:lip} estimates the selectivity of each filter using statistics from all previous batches, 
which can be inefficient for certain foreign key distributions in the fact table. 
Consider the case where some filter $f$ is very selective for the first $T$ batches and not selective for the remaining batches. 
(For example, a filter $f$ filtering for \texttt{year} $\geq 2017$ and the \texttt{Date} table is sorted in \texttt{year}.) In this case, LIP would obtain a good estimate of the selectivity of $f$ during the first $T$ iterations, and thus tend to apply $f$ early in the remaining iterations. However, it is more efficient to postpone applying $f$ in the remaining iterations, despite $f$ has good selectivity in the first $T$ iterations. One remedy to this is to only ``remember" the hit/miss statistics of each filter over the previous $k$ batches.

\begin{figure*}[h!]
	\centering
	\tikz\path (0,0) node[draw, text width=.8\textwidth, rectangle, inner xsep=20pt, inner ysep=10pt]{
		\begin{minipage}[t!]{\textwidth}
			{\sc Procedure}: \texttt{LIP-$k$}
			\\
			{\sc Input}: a fact table $F$ and a set of $n$ Bloom filters $f_i$ for each $D_i$ with $1 \leq i \leq n$
 			\\
			{\sc Output}: Indices of tuples in $F$ that pass the filtering
			\begin{tabbing}
				Aaa\=aaA\=Aaa\=Aaa\=Aaa\=AAAAAAAAAAAAAAAAAAAAAAAAA\=A \kill
				1.\> Initialize $I = \emptyset$
				\\
				2.\> {\bf foreach } filter $f$ {\bf do}
				\\
				3.\>\> Initialize $count[f] \leftarrow 0$, $pass[f] \leftarrow 0$ 
				\\
				4.\>\> Initialize $count\_queue[f]$ with $k$ zeros and $pass\_queue[f]$ with $k$ zeros.
				\\
				5.\> Partition $F = \bigcup_{1 \leq t \leq T}F_t$. 
				\\
				6.\> {\bf foreach } fact block $F_t$ {\bf do} 
				\\
				7.\>\> {\bf foreach } filter $f$ in order {\bf do}
				\\
				8.\>\>\> {\bf foreach} index $j \in F_t$ {\bf do}
				\\
				9.\>\>\>\> $count[f] \leftarrow count[f] + 1$
				\\
				10.\>\>\>\> {\bf if }$f$ contains $F_t[j]$ 
				\\
				11.\>\>\>\>\> $I \leftarrow I \cup \{j\}$ 
				\\
				12.\>\>\>\>\> $pass[f] \leftarrow pass[f] + 1$
				\\
				13.\>\>\> $count\_queue[f].dequeue()$ and $pass\_queue[f].dequeue()$
				\\
				14.\>\>\> $count\_queue[f].enqueue(count[f])$ and $pass\_queue[f].enqueue(pass[f])$
				\\
				15.\>\>\> Reset $count[f] \leftarrow 0$, $pass[f] \leftarrow 0$ 
				\\
				16.\>\> {\bf sort} filters $f$ in nondesending order of $sum(pass\_queue[f])/sum(count\_queue[f])$
				\\
				17.\> {\bf return } $I$
			\end{tabbing}  
		\end{minipage}
	};
	\caption{The LIP algorithm for computing the joins.}
	\label{fig:lip-k}
\end{figure*}

Empirical data shows that for the orignal SSB dataset and certain queries, LIP-$k$ is as fast as LIP, and for certain datasets and queries LIP-$k$ is faster than LIP. Detailed empirical data are presented and analyzed in Section \ref{sec:experiment}.



\subsection{Competitive Ratio Analysis}

The LIP strategy and its variant LIP-$k$ depicted in Figures \ref{fig:lip} and \ref{fig:lip-k} are \textit{deterministic}, i.e.\ multiple executions over the same fact table would produce the same result. Experimental results show that LIP has faster execution time compared to hash join in the optimal sequence \cite{zhu2017looking} on the benchmark dataset, in which the keys are distributed almost uniformly. However, it can be shown that any deterministic mechanism in the worst case can never achieve a competitive ratio less than $n$ when played against an adversary producing an adversarial dataset.


\begin{theorem}\label{thm:det-n}
	Let $n$ be the number of filters in the LIP problem. There is no deterministic mechanism $\mathcal{M}$ achieving a competitive ratio less than $n$ for the \textsc{LIP} problem.

\end{theorem}

\begin{proof}
	We present an adversary to the arbitrary mechanism $\mathcal{M}$ such that $\mathcal{M}$ only achieves a competitive ratio of $n$ in the worst case. 
	Let the $n$ filters be $f_1, f_2, \dots, f_n$, 
	and let $S_k$ denote the filter sequence that will be used to filter batch $k$. 
	Let each batch contain $m$ tuples.
	At each iteration $k$, prior to LIP's probe phase, the adversary observes $S_k$ and produces a batch of tuples $\{t_1, t_2, \dots, t_m\}$, where $t_i \in f_n$ but $t_i \notin f_j$ for any $j < n$. The adversary then feeds this batch into mechanism $\mathcal{M}$. Thus, $\mathcal{M}$ performs $n$ filter probes to eliminate each tuple, whereas the optimal sequence is to apply $f_n$ first. Thus, $\mathcal{M}$ achieves a competitive ratio of $n$.
\end{proof}

It might be possible to design a randomized mechanism that can achieve a better competitive ratio than $n$. The randomized mechanism would, at the end of each batch, select a sequence of applying the filters from a distribution of all filter permutations, based on the estimated selectivities. However, we have not obtained any algorithmic upper bound on the competitive ratio. Furthermore, a randomized approach creates a new concern of being too computationally heavy.

In the practical perspective however, one wishes to minimize the total running time of the mechanism, which is effectively the sum of the running time of the mechanism and the running time of building the filters and performing the probes. A trade-off between having a near optimal mechanism that consumes much time and allowing many failed probes to eliminate each non-participating tuple is therefore of much interest. 





\section{Database Implementation}

We have developed a prototype database system supporting basic select and join operations on top of Apache Arrow \cite{apachearrow}, a column-store format. This minimal prototype is sufficient to benchmark the performance of our Hash join, LIP, and LIP-$k$.

Our implementations of LIP and LIP-k only support left-deep join tree plans where the fact table is the ``outer table" in every join. We assume that the fact table schema contains foreign keys to all dimension tables, and each dimension table is single-key. Given a star schema fact table $F$ and dimension tables $D_i$ for $1 \leq i \leq n$, a join query in our system specifies selectors $\sigma_F$ for $F$ and $\sigma_i$ for each $D_i$, and executing that query will return 

$$\sigma_F(F) \JOIN \sigma_1(D_1) \JOIN \dots \JOIN \sigma_n(D_n)$$

\noindent projected on the schema of $F$, {\it i.e.}~we output the tuples in $F$ that can be joined with each $D_i$. The supported primitive selectors allow for selection ($=, \leq, \geq, <, >$) on scalar values and ranges. Range selections are executed using \texttt{BETWEEN} ($\ell$, $h$), which selects all $x$ with $\ell \leq x \leq h$. 

The selectors for each dimension can be either a primitive selector consisting of simple predicate ({\it e.g.} \texttt{ORDER DATE} = 1997) 
or a composition (logical AND/OR) of multiple primitive/composite selectors. 
This is implemented using the Composite design pattern.
Apache Arrow does not yet support vectorized string comparison operations nor vectorized range comparison operations.
For queries involving string and/or range selections, 
we instead scan along the column, checking which rows satisfy the selection predicate. 
Such selections are inherently slower than the supported vectorized selections. 
Because all of our join implementations must implement row-wise selection for string and range predicates, 
all algorithms suffer the same slowdown. 
Thus, this implementation caveat does not preclude us from studying the relative performance of LIP and LIP-$k$.

We only support string and numeric data types. All numeric data is stored as 64-bit integers by default.

The hash join algorithm first produces a hash table $T_i$ for each $\sigma_i(D_i)$, projected on the $k_i$, and then probe each tuple in the fact table against all $T_i$. We used Sparseepp (accessible at \url{https://github.com/greg7mdp/sparsepp}) as our implementation of the hash table, in which the sparsehash by Google (accessible at \url{https://github.com/sparsehash/sparsehash}) is used as the underlying hash function. All primary keys are regarded as 64-bit integers.

For LIP and LIP-$k$, the succinct filter structure we choose is the Bloom filters. 
The default false-positive rate is set to 0.001, which requires 10 hash functions.
We use Knuth's Multiplicative hash function, extended to accept a 64-bit integer as a seed. 
Our minimal implementation does not support selectivity estimation, 
so we initialize each Bloom filter assuming $\sigma(D_i) = \frac{1}{2}$, 
{\it i.e.} assuming half of the keys in each dimension table will be inserted into the filter.

Our code is available at \url{https://github.com/NicholasCorrado/CS764}.




\section{Database Implementation}

   \section{Empirical Results}\label{sec:experiment}

In this section we present the empirical results obtained from running Hash-join, LIP and LIP-$k$ on several datasets. In Section \ref{sec:dataset}, we present the dataset we used and how we manually apply skewness to generate skewed and adversarial datasets; in Section \ref{sec:time} we present the running time of multiple strategies and discuss their performance; and in Section \ref{sec:ratio} we discuss how $k$ affects the competitive ratio of LIP-$k$ empirically on our dataset.



\subsection{Skewed and Adversarial Datasets}
\label{sec:dataset}
The dataset for testing is obtained from the Star Schema Benchmark \cite{o2009star}. We will hard-code each queries in \cite{o2009star} to measure the join processing time.

We choose to skew the \texttt{ORDER DATE} foreign key in the \texttt{LINEORDER} table for simplicity in generating skew datasets. We skew on the predicate $\texttt{ORDER DATE} = 1997 \vee 1998$.
We generated several skew

Let $f_i^A$ denote the selectivity of a Bloom filter on key A after processing the $i^{th}$ batch. 
Let $\sigma_i^A$ denote the selectivity of a Bloom filter on key A on the $i^{th}$ batch alone. 

\subsubsection{\texttt{lineorder-date-50-50.tbl}}

Skewed datasets are generated by first generating the SSB dataset with $\text{SF} = 1$.
We then edit the columns we want to skew. 

The aim of the adversarial dataset is to force LIP to perform the maximum number of filter probes possible.
We now show how to construct such a dataset where $n = 2$.
Let subscripts denote the batch index and superscripts denote the key column (A or B).
We start with the first batch having
$\sigma_1^A = \frac{1}{2} - \varepsilon$ and $\sigma_1^B = \frac{1}{2} + \varepsilon$ where $0 < \varepsilon < \frac{1}{2}$. Then for all $j > 1$, we let

\begin{align*}
\sigma_j^A &= 
    \begin{cases}
    1 & \text{$j$ even} \\[0.5em]
    0 & \text{$j$ odd} \\
    \end{cases}\\[0.5em]
\sigma_j^B &= 
    \begin{cases}
    0 & \text{$j$ even} \\[0.5em]
    1 &  \text{$j$ odd} \\
    \end{cases}
\end{align*}

Thus, the optimal filter sequence $S^{OPT}$ is 

\begin{align*}
S^{OPT} &= 
    \begin{cases}
    (B, A) & \text{$j$ even} \\[0.5em]
    (A, B) & \text{$j$ odd} \\
    \end{cases}\\[0.5em]
\end{align*}

\DeclarePairedDelimiter\floor{\lfloor}{\rfloor}
After processing batch $j$,  we have $f_j^A = \frac{\frac{1}{2} - \varepsilon + \floor{\frac{j}{2}}}{j}$ and $f_j^B = \frac{\frac{1}{2} + \varepsilon + \floor{\frac{j}{2}}}{j}$  which can be rewritten as

\begin{align*}
f_j^A &= 
    \begin{cases}
    \frac{1}{2} + \frac{\frac{1}{2}-\varepsilon}{j} & \text{$j$ even} \\[0.5em]
    \frac{1}{2} - \frac{\varepsilon}{j} &  \text{$j$ odd} \\
    \end{cases}\\[0.5em]
f_j^B &= 
    \begin{cases}
    \frac{1}{2} - \frac{\frac{1}{2}-\varepsilon}{j} & \text{$j$ even} \\[0.5em]
    \frac{1}{2} + \frac{\varepsilon}{j} &  \text{$j$ odd} \\
    \end{cases}\\[0.5em]
\end{align*}

and thus, LIPs filter ordering will be

\begin{align*}
S &= 
    \begin{cases}
    (A, B) & \text{$j$ even} \\[0.5em]
    (B, A) & \text{$j$ odd} \\
    \end{cases}\\[0.5em]
\end{align*}

which is the reverse of $S^{OPT}$. 
Hence, after the first batch has been processed, LIP will have worst-case performance on all remaining batches.


\begin{center}
\begin{tabular}{ |>{\ttfamily}r|>{\ttfamily}c|l| } 
\hline
{\bf Dataset Name} & {\bf Skewed Foreign Keys} & {\bf Description of Skew} \\
\hline
\hline
lineorder-date-50-50.tbl& ORDER DATE & First 50 batches have $\sigma_{{\texttt{ORDER DATE}} = 1997 \text{ OR } 1998} = 1$ \\
& & Next 50 batches have $\sigma_{{\texttt{ORDER DATE}} = 1997 \text{ OR } 1998} = 0$ \\ 
\hline
lineorder-date-first-half.tbl& ORDER DATE & First $N/2$ batches have $\sigma_{{\texttt{ORDER DATE}} = 1997 \text{ OR } 1998} = 1$ \\
& & Next $N/2$ batches have $\sigma_{{\texttt{ORDER DATE}} = 1997 \text{ OR } 1998} = 0$, \\
& & where $N$ is the total number of batches in \text{LINEORDER}\\
\hline
lineorder-date-linear.tbl& ORDER DATE & $\sigma_{{\texttt{ORDER DATE}} = 1997 \text{ OR } 1998}$ increases linearly from 0 to 1 \\
& & across the \texttt{LINEORDER} table\\
\hline
lineorder-date-part-adversary.tbl& ORDER DATE & See Section~\ref{}\\
& &First batch has $\sigma_{{\texttt{ORDER DATE}} = 1997 \text{ OR } 1998} = \frac{1}{2} - \varepsilon$ \\
& & and $\sigma_{\texttt{PART KEY}} = \frac{1}{2} + \varepsilon$ \\
& PART KEY & Next 5 batches have $\sigma_{{\texttt{ORDER DATE}} = 1997 \text{ OR } 1998} = 0$ \\ 
\hline
\end{tabular}
\end{center}

The adversarial dataset is constructed to make SSB query 4.2 achieve near worst-case performance for LIP. 



\subsection{Execution Time}
\label{sec:time}



\subsection{Competitive Ratio}
\label{sec:ratio}

We ran LIP-$k$ on each skewed dataset we produced and compute the competitive ratio of each LIP-$k$ by taking the maximum of the competitive ratios achieved across all queries in all datasets. The results are depicted in Figure \ref{fig:cr}. 

\begin{figure}
    \centering
    \subfloat[]{
        \includegraphics[width=0.43\textwidth,keepaspectratio]{cr-k-uniform}
    }   
    \quad
    \subfloat[]{
        \includegraphics[height=0.32\textwidth,keepaspectratio]{cr-k-skewed}
    }
    \caption{The competitive ratios of LIP-$k$ against different $k$ values. We ran LIP-$k$ on uniform data and skewed (and adversarial) dataset to produce (a) and (b) respectively. The data point at $k = 200$ represents LIP (which is essentially LIP-$\infty$).}
    \label{fig:cr}
\end{figure}

When the keys in the fact table columns are distributed uniformly, the filters need not react to the local changes. Hence for LIP with higher $k$, it remembers more batches in the uniform data, therefore may produce relatively more accurate estimate of the selectivities than the LIP with lower $k$. Hence the competitive ratio would decrease (slightly) when $k$ increases, as depicted in Figure \ref{fig:cr}.  

%@TODO: Explain the adversarial case using epsilons
Figure \ref{fig:cr} (b) displays how an adversarial dataset can make LIP-$k$ and LIP perform inefficiently. In fact, each LIP-$k$ with even $k$ achieves an approximation ratio less than 2; LIP and LIP-$k$ with odd $k$ achieve an approximation ratio of almost 2 precisely at Query 3.2 in dataset \texttt{date-part-adversary}. Query 3.2 has two joins, and thus the performance of LIP-$k$ with even $k$ matches the worst case competitive ratio. By construction of \texttt{date-part-adversary}, the first batch has $\sigma^{1}_{1} = 1/2-\
\varepsilon$ and $\sigma^{1}_{2} = 1/2+\varepsilon$, and  each batch $B_{2i+1}$ is an \textit{odd} batch, with $\sigma^{2i}_{1} = 0$ and $\sigma^{2i}_{2} = 1$; and each batch $B_{2i}$ is an \textit{even} batch, with $\sigma^{2i}_{1} = 1$ and $\sigma^{2i}_{2} = 0$. 

When $i \leq k$, LIP and LIP-$k$ execute identically, since LIP-$k$ has not yet 'forgotten' any previous batches. We now consider the case where $i > k$, {\it i.e.} where LIP-$k$ has forgotten at least the first batch. For odd $k$, LIP-$k$'s selectivity estimates always contains one more odd (or even) batch than the other, and thus the estimated selectivity and filtering sequence are in favor of the majority batch type. This yields a false prediction of the next batch, yielding a competitive ratio of 2. For even $k$, LIP-$k$'s selectivity estimates contain an equal amount of even an odd batches,  and thus estimated selectivities remain the same ($1/2$ by construction) throughout the execution, and the filter sequence does not change. Thus for half batches it is optimal, and for the other half it is worse, resulting in a competitive ratio of \[ \frac{1 \times 1/2 + 2 \times 1/2}{1} = 1.5,\] as depicted in Figure \ref{fig:cr}. For LIP, since it remembers the statistics from the beginning, by construction it would also make the same decision as LIP-$k$ with odd $k$, losing at every batch. 

LIP performs poorly because it never forgets the first batch.
%@TODO: I will explain this better!!!!
\footnote{An astute observer might notice that for $k = 200$, the worst-case competitive ratio of $2$ is achieved even though $k$ is not odd. This is explained by recognizing that LIP-200 is essentially an approximation of LIP. LIP-200 performs just as poorly as LIP up until  the $201^{st}$ batch, after which it estimates both selectivities as 1/2.}



\begin{align*}
\varepsilon_i = 
    \begin{cases}
    \frac{\varepsilon}{2k+1} & \text{for odd $i$, where $i = 2k + 1$} \\ 
    \frac{\frac{1}{2} - \varepsilon}{2k} & \text{for even $i$, where $i = 2k$}
    \end{cases}
\end{align*}

If $k$ is even, then LIP-k will (after the first $k$ batches have been processed) have $\sigma^A_i = \sigma^B_i = \frac{1}{2}$. Thus,

%$k = 1, 2, 3, 4, 5, 10, 15, 20, 25, 30, 35, 40, 45, 50, 60, 70, 80, 90, 100$ on 


\section{Future Works and Concluding Remarks}

The contributions of this project come in two-folds: We implemented Hash-join and LIP on top of Apache Arrow and thus provided an interface for future integration of Hustle on Apache Arrow; and we proposed LIP-$k$, a variant of LIP, which opens up an area for improving the performance LIP. 
We find that for certain queries, 
LIP-$k$ performs better than LIP, 
while for other queries,
LIP performs better than LIP-$k$.

The budget allowed for filtering each tuple is limited --- roughly 100 CPU cycles. Hence, computationally-heavy approaches to maintaining filter statistics are not practically beneficial. Therefore, we will investigate dynamically sampling a segment of the fact table to have true estimates of each filters, and then stick to this estimate for the next few batches. This has low maintenance overhead, and once the adaptive policy for dynamic sampling is established, we expect that to have better practical performance than LIP and LIP-$k$. Note that this strategy is still deterministic, still prone to the worst case $n$ competitive ratio bound established by Theorem~\ref{thm:det-n}.  
It would be interesting to study LIP in the online algorithmic setting to design an efficient mechanism utilizing randomness 
that offers better worst-case guarantees without burning too many CPU cycles.
Of course, CPU efficiency and randomness are at odds in such an approach. 

As discussed in Section~\ref{sec:time}, sometimes the inherent cost of aiming for the optimal filter sequence outweighs its benefit.
In such cases, it is better to simply aim for ``good enough".
With this in mind, it would be interesting to develop a variant of LIP that strives for ``good enough" rather than optimality.


% Future work below

% In this work, there exists an adversarial fact table for all variants of LIP have that would force the algorithm to achieve a competitive ratio of $n$. However, reasonable applications of some existing random online algorithm mechanisms may yield a competitive ratio of $O(\log n)$ using the weighted majority algorithm \cite{littlestone1994weighted}. However, multiplicative updates on the weights may change the weighted average filter to use, but will not change the sequence of all filters if only comparing them based on their weights. More work is required here to either provide a mechanism to achieve a better competitive ratio, or provide a lower bound reduction showing that the competitive ratio of $n$ is tight even for random mechanisms.


\section*{Acknowledgements}

The authors wish to thank Prof.\ Jignesh Patel for providing constant feedback on this project and Kevin Gaffney for helping us with Apache Arrow specifics. The second author wishes to thank Prof.\ Paris Koutris for the suggestion of working on a practical project when the lemma production pipe is jammed. It works.


\bibliographystyle{plain}
\bibliography{rep}


\end{document}
